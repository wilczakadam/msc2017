\chapter{Operacje teoriomnogościowe}
\thispagestyle{chapterBeginStyle}

W tym rozdziale wprowadzimy pojęcie \textit{podobieństwa Jaccarda} zbiorów oraz przedstawimy algorytm \texttt{MinHash} służący do estymacji takiego podobieństwa zbiorów bez konieczności pamiętania wszystkich jego elementów.
 Następnie omówimy podstawowe metody estymacji liczności zbiorów powstałych w wyniku wykonania operacji teoriomnogościowych w przypadku, gdy nie są znane bezpośrednio zbiory na jakich wykonywane są te operacje, a jedynie ich szkice. 

\subsubsection{Podobieństwo Jaccarda}
Podobieństwo Jaccarda dwóch zbiorów definiujemy jako:
\begin{equation}
    J(A, B) = \frac{|A \cap B|}{|A \cup B|}.
\end{equation}
Definicję tę możemy także uogólnić  na $n$ zbiorów \cite{adroll}:
\begin{equation}
    J(A_1, A_2, ..., A_n) = \frac{|A_1 \cap A_2 \cap ... \cap A_n|}{|A_1 \cup A_2 \cup ... \cup A_n|}.
    \label{jacc_multi}
\end{equation}

\subsubsection{Algorytm MinHash}
\label{minhash}
W tym podrozdziale opiszemy działanie algorytmu \texttt{MinHash} oraz wyjaśnimy w jaki sposób możemy go zastosować do estymacji \textit{podobieństwa Jaccarda} zbiorów.
Algorytm \texttt{MinHash} korzysta z techniki tzw. \textit{min-wise hashing}. Schemat ten został po raz pierwszy zaproponowany przez Andrei Brodera \cite{broder}, jako narzędzie do określania podobieństwa stron internetowych.

Niech $h$ będzie funkcją haszującą.
%., która mapuje elementy ze zbiorów na liczby.
 %całkowite. 
 %\JL{czy to ma isitotne znaczenie, ze calkowite?}
 %\AW{chyba niekoniecznie - przetlumaczylem doslownie z "integer"}
 %\JL{Może tak jak teraz?}
 Dla dowolnego zbioru $X$ zdefiniujmy $h_{min}(X)$ jako minimalny element z $X$ względem funkcji $h$ , to jest taki element $x \in X$ dla którego $h(x)$ osiąga najmniejszą wartość.

Zastosujemy funkcję $h_{min}$ na zbiorach $A$ i $B$. Zakładając że nie wystąpią żadne kolizje haszy, otrzymamy dokładnie tę samą wartość, jeśli element $x$ należący do sumy zbiorów $A \cup B$ osiągający minimalną wartość $h(x)$ znajduje się także w przecięciu tych zbiorów $A \cap B$. Prawdopodobieństwo zajścia takiego zdarzenia jest równe właśnie indeksowi \textit{Jaccarda} tych zbiorów:
\begin{equation}
    Pr(h_{min}(A) = h_{min}(B)) = J(A, B)
    \label{jacc_prob}
\end{equation}
Łatwo zatem pokazać, że jeśli $I$ jest zmienną losową przyjmującą wartość $1$ gdy $h_{min}(A) = h_{min}(B)$ i $0$ w przeciwnym przypadku, wówczas $I$ jest nieobciążonym estymatorem dla $J(A, B)$ \cite{minhash}.

Ponieważ zmienna $I$ ma zbyt dużą wariancję, aby być przydatnym estymatorem indeksu \textit{Jaccarda}, główną ideą algorytmu \texttt{MinHash} jest zmniejszenie tej wariancji poprzez użycie estymatora będącego średnią z wielu niezależnych eksperymentów tego typu. Najprostsza wersja algorytmu zakłada użycie $k$ różnych funkcji haszujących, gdzie $k$ to ustalony parametr. Innymi słowy, każdy zbiór $X$ jest reprezentowany przez $k$ wartości $h^{(1)}_{min}(X), h^{(2)}_{min}(X), \ldots, h^{(k)}_{min}(X)$ policzonych przez $k$ funkcji haszujących.
Estymator $J(A, B)$ w tej wersji algorytmu wygląda następująco:
\begin{equation}
    \hat{J}(A, B) = \frac{1}{k}\sum_{i=1}^{k}[h_{min}^{(i)}(A) = h_{min}^{(i)}(B)],
    \label{jacc_est}
\end{equation}
gdzie $[x]$ jest notacją \textit{Iversona} dla zdarzenia $x$, zdefiniowaną jako 
$$[x] = \left\{ \begin{array}{rl}
	1 &\mbox{ jeśli zdarzenie $x$ jest prawdziwe, } \\
	0 &\mbox{ w p.p.}
\end{array} \right.$$ %\JL{uzupelnić}
Tak zdefiniowany estymator jest nieobciążonym estymatorem $J(A, B)$ \cite{minhash}, a jego wariancja wynosi:
\begin{equation}
    Var(\hat{J}(A,B)) = \frac{J(A, B)(1 - J(A, B))}{k}
    \label{jacc_var}
\end{equation}
Ten wzór można łatwo wyprowadzić. Zauważmy, że estymator $\hat{J}(A, B)$ jest zmienną losową będącą sumą $k$ niezależnych prób \textit{Bernoulliego} postaci $[h_{min}^{(i)}(A) = h_{min}^{(i)}(B)]$, a jak wiemy ze wzoru (\ref{jacc_prob}) prawdopodobieństwo sukcesu takiej jednej próby jest równe $J(A, B)$. Oznaczmy przez $X_i = [h_{min}^{(i)}(A) = h_{min}^{(i)}(B)]$. Wówczas możemy zapisać:
\begin{equation}
	\hat{J}(A,B) = \frac{1}{k}\sum_{i=1}^{k}X_i
\end{equation}
Nakładając wariancję otrzymujemy:
\begin{flalign}
	Var(\frac{1}{k}\sum_{i=1}^{k}X_i) = 
	\\
	\frac{1}{k^2}Var(\sum_{i=1}^{k}X_i) = 
	\\
	\frac{1}{k^2}\sum_{i=1}^{k}Var(X_i) =
	\\
	\frac{1}{k^2}\sum_{i=1}^{k}J(A, B)(1 - J(A, B)) =
	\\
	\frac{J(A, B)(1 - J(A, B))}{k}
	\label{jacc_var2}
\end{flalign}


Istnieje również drugi, mniej znany wariant algorytmu \texttt{MinHash}, który będziemy wykorzystywać w dalszej części pracy.
 Używa on analogicznego estymatora jak \ref{jacc_est}, ale korzysta z tylko jednej funkcji haszującej. Zamiast generować $k$ różnych funkcji haszujących, przechowujemy $k$ najmniejszych wartości dla porównywanych zbiorów $A_1, A_2$ i korzystamy z tylko jednej funkcji haszującej (analogicznie jak w algorytmie \texttt{MinCount}) \cite{adroll}. Posiadamy zatem zbiór haszy, który możemy traktować jako losową próbkę z $\bigcup A_i$. Następnie dla $k$ najmniejszych wartości z tej próbki zliczamy ile z nich znajduje się również wśród $k$ najmniejszych wartości w próbkach poszczególnych zbiorów. Całość dzielimy przez $k$ i otrzymujemy alternatywną wersję estymatora indeksu \textit{Jaccarda} \cite{adroll}:

 \begin{equation}
 \hat{J}(A_1, A_2) = \frac{|min_{k}\{min_{k} \{ h(A_1) \} \cup min_{k} \{ h(A_2) \} \} \cap (min_{k}\{h(A_1)\} \cap min_{k}\{h(A_2)\})|}{k}
 \end{equation}
 gdzie $h(A_i)$ oznacza zbiór zhaszowanych elementów zbioru $A_i$, a $min_{k}\{S\}$ jest funkcją zwracającą $k$ najmniejszych elementów ze zbioru $S$ (w naszym przypadku są to hasze).
 Estymator ten możemy łatwo uogólnić dla \textit{indeksu Jaccarda} $n$ zbiorów, zdefiniowanego w (\ref{jacc_multi}):
\begin{equation}
    \hat{J}(A_1, A_2, ..., A_n) = \frac{|min_{k}\{\bigcup min_{k} \{ h(A_i) \} \} \cap (\bigcap min_{k}\{h(A_i)\})|}{k}
\end{equation}

\section{Naiwna estymacja operacji teoriomnogościowych}
\label{naive_est}

Omówimy teraz naiwne metody wyników estymacji operacji teoriomnogościowych na szkicach, skupimy się na operacji sumy i przekroju. Dla niektóre szkiców danych operacja sumy może być zdefiniowana w stosunkowo naturalny sposób. Dla przykładu, łatwo zauważyć, że  oszacowanie liczności dla sumy dwóch zbiorów $A_1$ i $A_2$ przy użyciu szkiców $M_1$ i $M_2$ związanych z algorytmem \texttt{HyperLogLog} sprowadza się do stworzenia nowego szkicu $M_{A_1\cup A_2}$, takiego, że 
$$M_{A_1\cup A_2}[i] = \max(M_{A_1}[i], M_{A_2}[i]).$$ 

Najważniejszą własnością powyższej operacji jest to, że powstały szkic jest identyczny ze szkicem, który powstałby z sumy zbiorów wejściowych $A_1\cup A_2$ oraz fakt, że suma dwóch szkiców daje w wyniku nowy szkic. Oznacza to że operacja sumy jest zamknięta i pozawala m.in. na sumowanie ze sobą sekwencyjnie większej liczby szkiców. Tę operację dla algorytmu \texttt{HyperLogLog} omówimy jeszcze dokładniej w podrozdziale \ref{HLL_sum}.
Bardziej formalnie, możemy zdefiniować taką operację $\dot{\cup}$, że dla dwóch dowolnych zbiorów $A, B$:
\begin{equation}
    S(A_1) \dot{\cup} S(A_2) = S(A_1 \cup A_2),
\end{equation}
%\JL{zamienilem znaczek $\hat{\cup}$ na $\dot{\cup}$}
gdzie $S$ to funkcja generująca szkic danych dla zbioru.

W przeciwieństwie do operacji sumy, dla szkiców rozważanych przez nas algorytmów nie istnieje naturalna operacja przekroju. Istnieją metody umożliwiające
oszacowanie mocy przekroju, ale nie zwracają one w wyniku nowego szkicu, tak jak operacja sumy. Jedną z takich podstawowych metod jest zastosowanie zasady \textit{włączeń i wyłączeń}: 
\begin{equation}
    |A_1 \cap A_2| = |A_1| + |A_2| - |A_1 \cup A_2|
    \label{incexc}
\end{equation}
i skorzystanie z wcześniej wyznaczonej estymacji sumy do policzenia estymacji przekroju.
Metodę tę można również wykorzystać do oszacowania różnicy zbiorów:
\begin{equation}
    \begin{aligned}
        |A_1 \setminus A_2| = |A_1 \cup A_2| - |A_2|.
    \end{aligned}
\end{equation}
Estymatę przekroju można również policzyć korzystając z podobieństwa \textit{Jaccarda}. Poniżej podajemy zestawienie estymatorów operacji sum i przekroju zdefiniowanych zgodnie z naiwnym podejściem opisanym powyżej:
\begin{flalign}
        \hat{N}(S_1 \hat{\cup} S_2) &= \hat{N}(S(A_1 \cup A_2))\\
        \hat{N}(S_1 \hat{\cap} S_2) &= \hat{N}(S(A_1)) + \hat{N}(S(A_2)) - \hat{N}(S(A_1 \cup A_2)), 
\end{flalign}
a także z wykorzystaniem podobieństwa \textit{Jaccarda}:

\begin{flalign}
\hat{N_1}(S_1 \hat{\cap} S_2) &= \hat{J}(S(A_1), S(A_2))\hat{N}(S_1 \hat{\cup} S_2) \label{jacc_1} \\
        \hat{N_2}(S_1 \hat{\cap} S_2) &= \frac{\hat{J}(S(A_1), S(A_2))}{1 + \hat{J}(S(A_1), S(A_2))}(\hat{N}(S(A_1)) + \hat{N}(S(A_2))). \label{jacc_2}
\end{flalign}
Wzór (\ref{jacc_1}) możemy uzasadnić korzystając z następującej własności:
\begin{equation}
	J(A_1, A_2)|A_1 \cup A_2| = \frac{|A_1 \cap A_2|}{|A_1 \cup A_2|}|A_1 \cup A_2| = |A_1 \cap A_2|.
\end{equation}
Natomiast wzór (\ref{jacc_2}) wynika z następujących przekształceń:
\begin{flalign}
&\frac{J(A_1, A_2)}{1 + J(A_1, A_2)}(|A_1| + |A_2|) \\
&= \frac{\frac{|A_1 \cap A_2|}{|A_1 \cup A_2|}}{1 + \frac{|A_1 \cap A_2|}{|A_1 \cup A_2|}}(|A_1| + |A_2|) \\
&= \frac{|A_1 \cap A_2|}{|A_1 \cup A_2|}\frac{|A_1 \cup A_2|}{|A_1 \cup A_2| + |A_1 \cap A_2|}(|A_1| + |A_2|) \\
&= \frac{|A_1 \cap A_2|}{|A_1 \cup A_2| + |A_1 \cap A_2|}|A_1| + \frac{|A_1 \cap A_2|}{|A_1 \cup A_2| + |A_1 \cap A_2|}|A_2| \\
&= \frac{|A_1 \cap A_2|(|A_1| + |A_2|)}{|A_1 \cup A_2| + |A_1 \cap A_2|}. \\
\end{flalign}
Następnie korzystając z faktu (\ref{incexc}) otrzymujemy:
\begin{equation}
	\frac{|A_1 \cap A_2|(|A_1| + |A_2|)}{|A_1 \cup A_2| + |A_1 \cap A_2|} = |A_1 \cap A_2|.
\end{equation}

Powyższe metody estymacji nie są jednak zbyt dokładne. Zwłaszcza w przypadku przekroju, gdzie błąd jest mniej więcej proporcjonalny do wielkości sumy lub większego zbioru, a oczekiwalibyśmy błędu ograniczonego rozmiarem mniejszego zbioru \cite{ting}. Wskutek użycia powyższych formuł często dochodzi również do anomalii w wyniku których oszacowanie liczności jest ujemne. Dlatego w dalszej części pracy zajmiemy się analizą innych metod szacowania wyników operacji teoriomnogościowych, które dają dokładniejsze wyniki i pozbawione są tego rodzaju anomalii. Ponadto, przynajmniej w przypadku algorytmu \texttt{MinCount}, pozawalają na zdefiniowanie zamkniętej operacji przekroju.


% JL - chyba to jest niepotrzebne:
%%W tym rozdziale przyjrzymy się dokładniej naiwnym podejściom do aproksymacji operacji teoriomnogościowych dla szkiców danych zarówno dla algorytmu \texttt{MinCount} jak i \texttt{HyperLogLog}. Takie podejścia są łatwe w implementacji i nie wymagają żadnych modyfikacji istniejących algorytmów ale, zwłaszcza w przypadku przekroju i różnicy, są stosunkowo niedokładne, mogą prowadzić do anomalii oraz w przypadku niektórych algorytmów (np. \texttt{HyperLogLog}) nie pozwalają na zdefiniowanie zamkniętego operatora zwracającego nowy szkic a jedynie na estymację wyniku liczności.

\section{HyperLogLog}

\subsection{Operacja sumy}
\label{HLL_sum}

W poprzedniej sekcji wyjaśniliśmy, że algorytm \texttt{HyperLogLog} posiada naturalną, zamkniętą operację sumy teoriomnogościowej na swoich szkicach. Jest ona wyjątkowo nieskomplikowana i sprowadza się do znalezienia maksimum na każdym z rejestrów spośród sumowanych szkiców \cite{oertl}. Mając dwa szkice \texttt{HyperLogLog-a} rozmiaru $m$: $M_{A_1} = (M_{A_1}[1], ..., M_{A_1}[m])$ oraz $M_{A_2} = (M_{A_2}[1], ..., M_{A_2}[m])$ reprezentujące dwa zbiory $A_1$ oraz odpowiednio $A_2$, procedura tworząca szkic $M_{A_1 \cup A_2} = (M_{A_1 \cup A_2}[1], ..., M_{A_1 \cup A_2}[m])$ reprezentujący sumę $A_1 \cup A_2$ wygląda następująco:
\begin{equation}
    M_{A_1 \cup A_2} = \max(M_{A_1}[i], M_{A_2}[i]) \quad \textbf{dla} \quad i = 1, ..., m.
\end{equation}

Pamiętajmy, że w ten sposób możemy sumować szkice o tych samych parametrach $p$ i $q$ (patrz rozdział \ref{HLL_sketch}). Istnieje jednak możliwość sumowania dwóch szkiców o różnych parametrach parach $(p, q)$ oraz $(p',q')$, albowiem każdy szkic \texttt{HyperLogLog} opisany przez parę $(p, q)$ może zostać zredukowany do szkicu opisanego przez $(p', q')$, jeśli spełniony jest warunek $p' \leq p$ oraz $p' + q' \leq p + q$. Taka transformacja jest bezstratna, tj. powstały szkic jest taki sam jak szkic który powstałby przez dodawanie tych wszystkich elementów od początku do szkicu opisanego przez $(p', q')$ \cite{oertl}. Algorytm \ref{HLL_compress} przedstawia pseudokod procedury wykonującej taką transformację.
%\JL{opisac doklanie jak tak redukacja sie odbywa!
%\AW{Czy wystarczy pseudokod z pracki w której go znalazłem? Nawet tam nie jest dokładnie opisane co tu się dzieje, więc nie wiem czy jest sens sie nad tym rozpisywac - ten fakt o kompresji chcialem tutaj podac tak bardziej informacyjnie zeby bylo wiadomo ze mozna sumowac szkice o roznych p i q}
%\JL{Na razie, niech tak zostanie}
\begin{algorithm}
	\begin{algorithmic}
		\State $p, q, M $ - parametry oraz kolekcja rejestrów kompresowanego szkicu
		\State $p', q' $ - parametry wejściowe szkicu docelowego
		\State Założenie: $ p' \leq p$ oraz $p' + q' \leq p + q$
		\State $Q(s) $ -  funkcja zwracająca pozycję pierwszej jedynki w ciągu bitów $s$ 
		\newline
		\Function{Compress}{$M$}
		\State $m' = 2^{p'}$
		\For {$i = 1 .. m'$}
			\State $M'[i] \gets 0$
		\EndFor
		\For {$i = 1 .. m'$}
			\State $b \gets (i - 1)2^{p-p'}$
			\State $j \gets 1$
			\While {$j \leq 2^{p-p'}$ and $M[b+j] = 0$}
				\State $j \gets j + 1$
			\EndWhile
			\If {$j = 1$}
				\State $M'[i] \gets \min(M[b+j] + (p - p'), q' + 1)$
			\ElsIf {$j \leq 2^{p-p'}$}
				\State ${{\langle}e_1, e_2, ..., e_{p-p'}{\rangle}}_2 \gets j - 1$
				\State $v \gets {{\langle}e_1, e_2, ..., e_{p-p'}{\rangle}}_2$ 
				\State $M'[i] \gets Q(v)$
			\Else
				\State $M'[i] = 0$
			\EndIf
		\EndFor
		\State \Return $M'$
		\EndFunction
		
	\end{algorithmic}
	\caption{Procedura kompresji szkicu \texttt{HyperLogLog}}
	\label{HLL_compress}
\end{algorithm}

\subsection{Operacja przekroju i różnicy}

Operacja sumy na szkicach algorytmu \texttt{HyperLogLog} sprawdza się całkiem dobrze i jest  łatwa w implementacji. Niestety nie są znane naturalne i zamknięte operacje przekroju oraz różnicy na tych szkicach. Oczywiście stosunkowo łatwo możemy estymować liczność przekroju zbiorów korzystając z zasady włączeń i wyłączeń lub podobieństwa Jaccarda.

O ile metoda oparta na zasadzie włączeń i wyłączeń nie sprawdza się w tym przypadku i prowadzi do anomalii, to metoda wykorzystująca podobieństwo \textit{Jaccarda} (\ref{jacc_1}) do estymacji przekroju jest stosunkowo dokładna, co potwierdzają wykonane przez nas eksperymenty. Wymaga ona jednak dodatkowej struktury danych pozwalającej na estymację indeksu \textit{Jaccarda} zbiorów. Częstą praktyką jest wykorzystanie do tego algorytmu \texttt{MinHash} \cite{adroll}. Wyniki związane z tym podejściem zostaną jeszcze omówione w kolejnym rozdziale.


\section{MinCount}
\label{impr_mincount}

\subsection{Operacja sumy}
\label{impr_sum}

Przyjrzyjmy się teraz operacji sumy dla algorytmu \texttt{MinCount}. Załóżmy, że posiadamy dwa szkice danych $S_1 = (H_1, {\tau}_1)$ oraz $S_2 = (H_2, {\tau}_2)$ dla zbiorów $A_1$ i $A_2$. Chcemy otrzymać szkic $(H_u, {\tau}_u)$ zbioru $A_1 \cup A_2$. W tym przypadku naturalne wydaje się być wykonanie sumy zbiorów haszy $H_1$ oraz $H_2$ i utworzenie z nich nowego zbioru $H_u$ zawierającego, $k$ najmniejszych haszy z sumy $H_1 \cup H_2$. Takie podejście  odrzuca jednak dużą ilość informacji. Weźmy jako przykład dwa zbiory $A_1$ oraz $A_2$, które są rozłączne i o tej samej liczności.
Posiadając $2k$ haszy ze zbiorów $H_1$ i $H_2$ odrzucamy połowę z nich - prowadzi to do nawet dwukrotnego zwiększenia wariancji estymatora. 
 Najlepszym estymatorem byłby w tym przypadku $\hat{N}(S_1 \cup S_2) = \hat{N}(S_1) + \hat{N}(S_2)$. Jego wariancja - zgodnie ze wzorem (\ref{mincount_var}) wynosi $$Var(\hat{N}(S_1 \cup S_2)) \approx \frac{|A_1|^2}{k} + \frac{|A_2|^2}{k} = \frac{|A_1|^2 + |A_2|^2}{k}.$$ Ponieważ zbiory $A_1$ i $A_2$ posiadają tyle samo elementów, możemy skorzystać z tożsamości $x^2 + x^2  = \frac{(2x)^2}{2}$ oraz faktu, iż zbiory $A_1$ oraz $A_2$ są rozłączne, aby przedstawić wariancję w postaci $$\frac{|A_1|^2 + |A_2|^2}{k} = \frac{|A_1 \cup A_2|^2}{2k}.$$ Jednak pamiętajmy, że nasza dotychczasowa operacja sumy odrzuca $k$ haszy co powoduje dwukrotny wzrost wariancji do $\frac{|A_1 \cup A_2|^2}{k}$.
%\JL{OK. Tylko mam taką uwagę, żeby w całej pracy po wzorach dawać kropki lub przecinki, tak jakby to były normalne zdania. To jest w dobrym tonie. I najlepiej jeszcze odwołania do wzorów brać w nawiasy, tzn. ''ze wzoru (3.1) mamy''.}

W pracy \cite{ting} zaproponowana została modyfikacja, która zamiast ograniczać nowo powstały szkic do $k$ wartości, tworzy największy możliwy szkic, dzięki czemu nie następuje utrata informacji.
Oznaczmy przez ${\tau}(S)$ największy przechowywany hasz w szkicu $S$, przez $h(S)$ zbiór haszy przechowywanych w szkicu $S$ oraz $h(S, \tau)$ niech będzie zbiorem haszy w szkicu $S$ których wartości są mniejsze bądź równe $\tau$. Nowy operator sumy $\dot{\cup}$ na szkicach \texttt{MinCount} wygląda następująco:
%\JL{wprowadzić osobne oznaczenie $\dot{\cup} $ dla sumy na szkicach, rozumiem, że poniżej $S_1$ i $S_2$ to są szkice a nie zbiory haszy?}
\begin{flalign}
        {\tau}_{min} = \tau(S_1 \dot{\cup} S_2) = min({\tau}_1, {\tau}_2) \\
        h(S_1 \dot{\cup} S_2) = h(S_1, {\tau}_{min}) \cup h(S_2, {\tau}_{min})
\end{flalign}

W tej modyfikacji algorytmu odrzucamy te wartości które są większe niż wartość graniczna ${\tau}_{min}$, a szkice są następnie łączone poprzez sumę teoriomnogościową na zbiorach pozostałych haszy.

Zauważmy, że powstały szkic jest identyczny ze szkicem wielkości $|h(S_1 \cup S_2)|$ skonstruowanym z elementów ze zbioru $A_1 \cup A_2$. Naturalnie, estymator liczności dla tak stworzonego szkicu zdefiniowany jest następująco:
\begin{equation}
    {\hat{N}}_{impr}(S_1 \dot{\cup} S_2) = \frac{|h(S_1 \dot{\cup} S_2)| - 1}{{\tau}_{min}}.
\end{equation}

\subsection{Operacja przekroju}
\label{impr_inter}

Rozpatrzmy teraz operację przekroju dla algorytmu \texttt{MinCount}. W naiwnym podejściu wykorzystujemy zasadę włączeń i wyłączeń, jak zostało to pokrótce opisane w sekcji \ref{naive_est}. Podejście to jednak nie pozwala nam na zdefiniowanie zamkniętego operatora przekroju, a jedynie na wyznaczenie estymacji liczności przekroju.
Ponadto, takie podejście prowadzi zazwyczaj do dużych błędów estymacji, a także amonali w postaci ujemnych wartości estymacji.

Możemy jednak zdefiniować operator przekroju w podobny sposób, jak zdefiniowaliśmy poprawiony operator sumy w poprzednim podrozdziale \ref{impr_sum}. Tak jak poprzednio, zamiast szkicu ograniczonego do $k$ wartości, tworzymy największy możliwy szkic. Nowy operator przekroju na szkicach \texttt{MinCount} może być zdefiniowany następująco \cite{ting}:
\begin{flalign}
        {\tau}_{min} = \tau(S_1 \dot{\cup} S_2) = \tau(S_1 \dot{\cap} S_2) = min({\tau}_1, {\tau}_2), \\
        \label{sketch-cut}
        h(S_1 \dot{\cap} S_2) = h(S_1, {\tau}_{min}) \cap h(S_2, {\tau}_{min}).
\end{flalign}
Jedną z największych zalet takiej definicji jest fakt, że otrzymujemy tutaj operator zamknięty, czyli taki który w wyniku operacji przekroju na dwóch szkicach daje w wyniku nowy szkic. Estymator liczności dla tak zdefiniowanej operacji przekroju ma postać:
\begin{equation}
    {\hat{N}}_{impr}(S_1 \dot{\cap} S_2) = \frac{|h(S_1 \dot{\cap} S_2)| - \alpha(S_1, S_2)}{{\tau}_{min}}.
\end{equation}
Pamiętajmy, że aby zachować właściwą postać wzoru estymatora (\ref{mincount_est}) jedynkę w liczniku musimy odjąć tylko jeśli $\tau_{min}$ znajduje się w przekroju szkiców. Stąd we wzorze pojawia się funkcja $\alpha(S_1, S_2)$ zdefiniowana jako:
%\JL{To zdanie jest jakos nieskladne, postac jakiego wzoru chcemy zachowac?}
%\AW{Poprawilem nieco to zdanie}
$$\alpha(S_1, S_2) = \left\{ \begin{array}{rl}
 1 &\mbox{ jeśli ${\tau}_{min} \in h(S_1 \dot{\cap} S_2)$}, \\
  0 &\mbox{ w p.p.}
       \end{array} \right.$$
%\JL{Tutaj zdanie wyjasnienia z czego wynika odejmowanie tej alfy}       
       
       \section{Estymacja metodą największej wiarygodności}
       
       Przyjmijmy, że dysponujemy szkicami dla pewnych zbiorów $A_1$ i $A_2$. Estymacja liczności dla zbiorów $A_1 \cup A_2$ czy 
       $A_1 \cap A_2$ w oparciu o te szkice, jest problemem estymacji parametru rozkładu \cite{ting}. Rozsądnym może więc wydawać się wyprowadzenie estymatorów dla wyników operacji teoriomnogościowych metodą \textit{największej wiarygodności} (NW).
       Estymator wyprowadzony tą metodą ma szereg istotnych własności. Wraz ze wzrostem rozmiaru próby jest asymptotycznie nieobciążony, asymptotycznie najefektywniejszy, ma asymptotycznie rozkład normalny, a także jest zgodny. Estymator $T_n$ parametru $\theta$ nazywamy \textit{zgodnym} jeśli ciąg $\{T_n\}$ \textit{zbiega według prawdopodobieństwa} 
        do prawdziwej wartości parametru gdy $n \rightarrow \infty$, czyli inaczej mówiąc - jeśli spełniona jest równość:
       \begin{equation}
     		\lim_{n \to \infty} Pr[|T_n - \theta| < \epsilon] = 1
     		\label{consistent_def}
       \end{equation}
       dla każdego $\epsilon > 0$.
       %\AW{może być cos takiego? taką definicję udalo mi sie znalezc i wyglada najsensowniej}
       %\JL{Zgrubsza OK, tylko chyba nie estymator $T_n$ zbiega, a ciag estymatorow $(T_n)_{n\geq1}$ zbiega, prosze zobaczyc np na angielska wikipedie}
       %\AW{updated}
       %\JL{OK}
%      
%       Estymator nazywamy zgodnym jeśli prawdopodobieństwo, że jego wartość będzie bliska wartości szacowanego przez niego parametru, wzrasta wraz z podniesieniem liczności próby \cite{cons}.
      
      W pracach \cite{ting} oraz \cite{oertl} podjęto wysiłki
      by wyprowadzić estymator metodą NW dla operacji teoriomnogościowych na szkicach odpowiednio dla algorytmu \texttt{MinCount} oraz \texttt{HyperLogLog}.
      Autor pracy \cite{ting} zwraca jednak uwagę, że
      w praktyce implementacja takiego estymatora na szkicach \texttt{MinCount} może okazać się bardzo kłopotliwa i opisuje tzw. \textit{estymator ważony}, który ma  mieć podobne własności statystyczne, a którego implementacja jest znacznie prostsza. 
  
 Podobnie jak w przypadku algorytmu \texttt{MinCount}, estymator NW operacji teoriomnogościowych na szkicach \texttt{HyperLogLog} opisany w 
 \cite{oertl} w praktyce może być dosyć trudny do efektywnego zaimplementowania, między innymi ze względu na konieczność maksymalizacji wielowymiarowej funkcji.
       
        W dalszej części naszej pracy omówimy  \textit{metodę estymatora ważonego} w kontekście algorytmu 
        \texttt{MinCount}. 
        Estymatory wyprowadzone tą metodą powinny był łatwo implementowalne, mieć niemal te same własności, jak estymatory wyprowadzone metodą \textit{estymatora największej wiarygodności} oraz łatwo uogólniać się na operacje teoriomnogościowe na większej liczbie zbiorów \cite{ting}.         
        Rozważymy zatem także  możliwość uogólnienia tej metody  dla szkiców algorytmu \texttt{HyperLogLog}.
        
        

       
       
       
